\pagenumbering{arabic}

\section{数据分析}

 	\subsection{全局分析}
	图\ref{fig:all}展示的是信号强度在二位地理位置上的分布.
	图中的四个红色的框分别代表教一, 主楼, 广场, 教二.
	通过右边的colormap可以直观的感受到信号在地理分布上的规律.
	\addfig[0.8]{all.png}{fig:all}{全局分析}  

 \indent	从中可以看到如下事实
	  \begin{enumerate}
		\item 对于教二和教一来说, 南边的信号要比北边要强, 猜测信号源在学校的北边
		\item 广场的信号分布相对均匀一些, 猜测是因为广场空间宽阔
		\item 教一南到广场北和广场南到教二北有明显衰减, 猜测是因为广场周边的树造成的
	  \end{enumerate}
	
 	\subsection{南北分析}
	为了分析信号在南北方向的分布规律, 我们把所有点在东西方向去平均, 得到图\ref{fig:sourth-north}
	图中"oneS"代表教一南, "squareN"代表广场北
	\addfig[0.8]{sorth-north.png}{fig:sourth-north}{南北分析}

 \indent	从中可以看到如下事实
		\begin{enumerate}
		  \item 对于一个实体(广场, 教一, 教二), 南面信号比北面强, 猜测是因为信号源在学校南方
		  \item	教二北比广场南的信号要弱, 这是唯一一个违背南强北弱规律的, 暂时无法解释
		\end{enumerate}

 \indent	求平均后的数据如表\ref{table:sourth-north-mean}
	\begin{table}[htbp]
\centering
\begin{tabular}{lllllll}
 \toprule
   & twoS    & twoN    & squareS & squareN & oneS    & oneN   \\ 
 \midrule
数据 & -58.732 & -59.688 & -57.94  & -58.968 & -59.746 & -61.19 \\ 
\bottomrule
\end{tabular}
\caption{南北方向平均值}
\label{table:sourth-north-mean}
\end{table}


 \indent	根据衰落计算公式$$\Delta P = \frac{1}{N} \sum _{i=1} ^{N} P_i^{outside} -\frac{1}{M} \sum _{j=1} ^{M} P_j^{inside}$$可以得到表\ref{table:sorth-north-delta}
	\begin{table}[htbp]
\centering
\begin{tabular}{ll}
\toprule 
                & 衰落损耗   \\ 
\midrule
twoS-twoN       & 0.965  \\     
towN-squareS    & -1.748 \\     
squareS-squareN & 1.028  \\     
squareN-oneS    & 0.778  \\     
oneS-oneN       & 1.444 \\
\bottomrule
\end{tabular}
\caption{南北方向衰落损耗}
  \label{table:sorth-north-delta}
\end{table}


 	\subsection{东西分析}
	为了分析信号在东西方向的分布规律, 我们把所有的点在南北方向上取平均, 得到图\ref{fig:west-east}
	图中x轴代表自西像东的第x个点
	\addfig[0.8]{west-east.png}{fig:west-east}{东西分析}
\\\indent	计算其统计特性可以得到表\ref{table:west-east-statistic}
	\begin{table}[htbp]
\centering
\begin{tabular}{lll}
  \toprule
       & mean       & std      \\
	 \midrule
result & -59.601250 & 2.126018\\
\bottomrule
\end{tabular}
\caption{东西方向的统计特性}
  \label{table:west-east-statistic}
\end{table}

	

